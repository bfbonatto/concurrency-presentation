% Intended LaTeX compiler: pdflatex
\documentclass[bigger]{beamer}
\usepackage[utf8]{inputenc}
\usepackage[T1]{fontenc}
\usepackage{graphicx}
\usepackage{longtable}
\usepackage{wrapfig}
\usepackage{rotating}
\usepackage[normalem]{ulem}
\usepackage{amsmath}
\usepackage{amssymb}
\usepackage{capt-of}
\usepackage{hyperref}
\usetheme{default}
\author{Bruno Bonatto}
\date{\today}
\title{Notes and Patterns in Concurrent Channel-Based Programming}
\hypersetup{
 pdfauthor={Bruno Bonatto},
 pdftitle={Notes and Patterns in Concurrent Channel-Based Programming},
 pdfkeywords={},
 pdfsubject={},
 pdfcreator={Emacs 28.2 (Org mode 9.6)}, 
 pdflang={English}}
\begin{document}

\maketitle

\section{Introduction}
\label{sec:orgdd19380}

\begin{frame}[label={sec:orgc68c724}]{Introduction}
Concurrent programming is a very difficult problem, fortunately many incredibly
intelligent people have spent many hours developing tools and theories to help
us develop correct and secure concurrent programs. This is a collection of notes
on some modern approaches to concurrent programming.
\end{frame}

\begin{frame}[label={sec:org6985a57}]{Motivational Example}
\href{https://github.com/bfbonatto/concurrency-presentation/blob/master/sieve.go}{Concurrent Sieve}
\end{frame}

\section{CSP}
\label{sec:org2b15e1b}

\begin{frame}[label={sec:org3fae2ae}]{Communicating Sequential Processes}
The basis for Go.
\end{frame}

\section{Pi Calculus}
\label{sec:org57debb8}

\begin{frame}[label={sec:orgce0476f}]{Pi Calculus}
A formal system for computation, analogous to the lambda calculus, focused
on concurrency and message passing.
\end{frame}

\begin{frame}[label={sec:orgd008e4d}]{Core Language}
Only two kinds of entities: processes and channels.

\begin{itemize}
\item \(\bar{x}y\) sends the value \(y\) along the port/channel \(x\).
\item \(x(z)\) receives a value from \(x\) and assigns it to the variable \(z\).
\item \(e_1\cdot e_2\) composes \(e_1\) and \(e_2\) sequencially.
\item \(e_1 | e_2\) composes \(e_1\) and \(e_2\) parallelly
\end{itemize}
\end{frame}

\begin{frame}[label={sec:orgec64900}]{Example}
\begin{equation*}
\bar{x}y \cdot e_1 \ |\ x(z) \cdot e_2 \ \Rightarrow \ e_1 \ |\  \{z \mapsto y\} e_2
\end{equation*}

This ``program'' sends the value \(y\) through the channel \(x\) from one process
to another, the sending process then continues to execute the program \(e_1\);
while the receiving process executes the program \(e_2\) but with all instances
of the variable \(z\) replaced by the value \(y\).
\end{frame}

\begin{frame}[label={sec:org569e2fe}]{Core Language}
Fresh channels can be introduced with \(v\).

The expression \((vx)e\) creates a fresh channel \(x\) in the scope \(e\).
For example:

\begin{equation*}
(vx)(\bar{x}\cdot e_1\ |\ x(z)\cdot e_2)
\end{equation*}

Localizes the channel \(x\), ensuring that no other processes can interfere with
communication on \(x\).
\end{frame}

\section{Go}
\label{sec:orgec5fc79}

\begin{frame}[label={sec:orge99e31e},fragile]{Common Concurrency Errors}
 \begin{block}{Channel safety}
Once a channel is closed, receive actions always succeed, but all send
and close actions raise a runtime error.
\end{block}

\begin{block}{Global deadlocks}
The Go runtime contains a \emph{global} deadlock detector that signals
a runtime error when \emph{all} goroutines in a program are stuck. However
it is often the case that, when certain libraries are imported (such as
the commonly used \texttt{net} package), the global deadlock is silently \emph{disabled}.
\end{block}

\begin{block}{Partial deadlocks}
Sometimes called \emph{liveness} failures, partial deadlocks occur when a program's
communication cannot progress despite some of its goroutines not being stuck.
\end{block}
\end{frame}

\begin{frame}[label={sec:org4c9a9c2}]{Partial Deadlock Example}
\href{https://github.com/bfbonatto/concurrency-presentation/blob/master/partial-deadlock.go}{Example}
\end{frame}

\section{Multiparty Session Types}
\label{sec:orge7938df}

\begin{frame}[label={sec:org6456fce}]{Multiparty Session Types}
TODO
\end{frame}

\section{References}
\label{sec:orge5100f7}

\begin{frame}[label={sec:orgb4ef151}]{References}
\begin{itemize}
\item Hoare (1978)
\item Pierce, Turner (2000)
\item Lange, Ng, Toninho, Yoshida (2017)
\item Lange, Ng, Toninho, Yoshida (2018)
\item Castro, Hu, Jongmas, Ng, Yoshida (2019)
\end{itemize}
\end{frame}
\end{document}